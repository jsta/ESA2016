\documentclass[compress,noflama,nosectionpages]{beamer}
%--------------------------------------------------------------------------
% Common packages
%--------------------------------------------------------------------------

\usepackage[normalem]{ulem}%jsta added
\usepackage{hyphenat}%jsta added
\usepackage{rotating}%jsta added
\usepackage{graphicx}
\usepackage{multicol}
\usepackage{tabularx,ragged2e}
\usepackage{booktabs}
\usepackage{listings}
\lstset{ %
language=[LaTeX]TeX,
basicstyle=\normalsize\ttfamily,
keywordstyle=,
numbers=left,
numberstyle=\tiny\ttfamily,
stepnumber=1,
showspaces=false,
showstringspaces=false,
showtabs=false,
breaklines=true,
frame=tb,
framerule=0.5pt,
tabsize=4,
framexleftmargin=0.5em,
framexrightmargin=0.5em,
xleftmargin=0.5em,
xrightmargin=0.5em
}

%--------------------------------------------------------------------------
% Load theme
%--------------------------------------------------------------------------
\usetheme{hsrm}


%------jsta test
% \setbeamertemplate{navigation symbols}{}
%  \makeatletter\
% \setbeamertemplate{footline}
% {
%  \pgfuseshading{beamer@barshade}
%  \ifbeamer@sb@subsection
%  \vskip-9.75ex
%  \else
%  \vskip-7ex
%  \fi
%  \begin{beamercolorbox}[ignorebg,ht=2.25ex,dp=3.75ex]{section in head/foot}
%  \insertnavigation{\paperwidth}
%  \end{beamercolorbox}
%  \ifbeamer@sb@subsection
%  \begin{beamercolorbox}[ignorebg,ht=2.125ex,dp=1.125ex,leftskip=.3cm,rightskip=.3cm plus1fil]{subsection in head/foot}
%  \usebeamerfont{subsection in head/foot}\inserctsubsectionhead
%  \end{beambercolorbox}
%  \fi
%  }
%  \setbeamertemplate{headline}{
%  \hskip1em\usebeamercolor[fg]{navigation symbols dimmed}
%  \insertslidenavigationsymbol
%  \insertframenavigationsymbol
%  \insertsectionnavigationsymbol
%  \insertsubsectionnavigationsymbol
%  \insertdocnavigationsymbol
%  \insertbackfindforwardnavigationsymbol
%  }
% \makeatother


%------

%jsta commented out%\usepackage{dtklogos} % must be loaded after theme
\usepackage{tikz}
\usetikzlibrary{mindmap,backgrounds}

%--------------------------------------------------------------------------
% General presentation settings
%--------------------------------------------------------------------------
\title{\nohyphens{Fine-scale spatial}}
\subtitle{\nohyphens{patterning of phytoplankton abundance in a coastal estuary}}
\date{August 2016}
\author{{\Medium Joseph Stachelek}, Christopher Madden, Stephen P. Kelly, Michelle Blaha}
\institute{South Florida Water Management District\\ {\Medium Everglades Division}}

%--------------------------------------------------------------------------
% Notes settings
%--------------------------------------------------------------------------
\setbeameroption{show notes}

\begin{document}
%--------------------------------------------------------------------------
% Titlepage
%--------------------------------------------------------------------------

\maketitle

%\begin{frame}[plain]
%  \titlepage
%\end{frame}


%--------------------------------------------------------------------------
% Content
%--------------------------------------------------------------------------
\section{Introduction}
\subsection{Introduction}
\begin{frame}{How can we describe phytoplankton distributions?}

\includegraphics[height=6cm,keepaspectratio=true]{images/landsat_border.png}\\

\end{frame}

\begin{frame}{The Discrete Approach}
	\begin{columns}
		\begin{column}[t]{4cm}
			\includegraphics[height=3.5cm,clip=true,trim = 0mm 0mm 0mm 0mm,keepaspectratio=true]{figures/fbmap_dflow.png}%
		\end{column}
		
		\begin{column}[c]{4cm}
			\includegraphics[height=3.5cm,clip=true,trim = 0mm 0mm 0mm 0mm,keepaspectratio=true]{figures/fbmap_wqmn.png}%
		\end{column}
		
		\begin{column}[b]{4cm}
			\includegraphics[height=3.5cm,clip=true,trim = 0mm 0mm 0mm 0mm,keepaspectratio=true]{figures/fbmap_wqmn.png}%
		\end{column}
	\end{columns}
\end{frame}

\begin{frame}{What do we miss with the Discrete Approach?}

	\begin{columns}
		\begin{column}[c]{6cm}
	  	\includegraphics[height=3.9cm,clip=true,trim = 0mm 0mm 0mm 0mm,keepaspectratio=true]{images/scipy_border.png}%
	  	\vspace{3pt}
	  	\begin{itemize}
	  		\item{Spatial Gradients}
	  	\end{itemize}
		\end{column}
		\begin{column}{6cm}
			\includegraphics[height=3.9cm,keepaspectratio=true]{images/pointsource_border.png}%
			\vspace{3pt}
			\begin{itemize}
	  		\item{Point Source Extent}
	  	\end{itemize}
		\end{column}
	\end{columns}
     
\end{frame}


\begin{frame}[label=data]{An Alternative - The Underway Approach}
  \begin{columns}
    \begin{column}{5.6cm}
      \begin{itemize}
        \item{Quarterly surveys}\vspace{15pt}\\
        \item{Measurements every 50m}\vspace{15pt}\\         \item{Emphasis on freshwater discharge}
      \end{itemize}
    \end{column}
    \begin{column}{6.8cm}
    	\begin{picture}(60,60)(0,-18)
      	\includegraphics[width=6.5cm,keepaspectratio=true]{figures/sm-figure0.png}
      \end{picture}
      \begin{picture}(80,80)(45,80)
      	\includegraphics[width=5.5cm,keepaspectratio=true]{images/20150212_110309.jpg}
      \end{picture}
    \end{column}
  \end{columns}
\end{frame}

\begin{frame}{Chlorophyll Modelling\textsuperscript{1,2}}
\vspace{2pt}
	\begin{columns}
		\begin{column}{5cm}
			{\footnotesize
			\begin{tabular}{| l | l |}
				\hline
				Instrument Package & Parameter \\ \hline
				Optical 1 & CDOM \\
				... & Chlorophyll \\ \hline
				Optical 2 & CDOM \\ 
				... & Chlorophyll \\
				... & Phycocyanin \\
				... & Phycoerytherin \\ \hline
			\end{tabular}
			}
			\end{column}
			\begin{column}{4.5cm}
				\includegraphics[width=4.2cm,keepaspectratio=true]{figures/chl_onetoone.png}
			\end{column}
	\end{columns}
	\small
	\texttt{DataflowR::chlcoef(201509, corcut = 0.75)}\\
	\texttt{> Initial correlation matrix}\\
	\texttt{> MLR with all variables...}\\
	\texttt{> Checking for redundancy in variables pairs}\\
	\texttt{> Generate AIC for candidates}\\
	\texttt{> Checking VIF...} 
	
\tiny{\phantom{\footnote{Sepp\"{a}l\"{a} et al. 2007 \textsuperscript{2}Venables and Ripley 2002}}}
\end{frame}

\begin{frame}{Spatial Modelling}

Inverse Path Distance Weighting (IPDW)

 * A non-model based machine-learning approach focused on out-of-sample prediction skill

As the crow flies or as the fish swims?\footnote{Little et al. 1997}\footnote[frame]{Suominen et al. 2010}
% 	\begin{columns}
%   	\begin{column}{6.8cm}
%     	\includegraphics[width=6.8cm,keepaspectratio=true,clip=true,trim= -0mm 0mm 0mm 0mm]{sm-figure1.png}
% 		\end{column}
% 		
% 		\begin{column}{5.4cm}
%     	\includegraphics[width=4.8cm,keepaspectratio=true]{Picture1.png}
%     	\begin{align*}
%     		V = \frac{\sum\limits_{i=1}^n v_i \frac{1}{d_i^p}}{\sum\limits_{i=1}^n \frac{1}{d_i^p}}
%     	\end{align*}
%   	\end{column}
% 
% \end{columns}
\end{frame}


\section{Results}
	\subsection{Results}

		\begin{frame}{Results}
			\includegraphics[height=6cm,keepaspectratio=true]{figures/chltimeseries.png}\\
		\end{frame}

		\begin{frame}{Results}
			\includegraphics[height=6cm,keepaspectratio=true]{figures/multipanel.png}\\
		\end{frame}

\section{Resources}
\subsection{Resources}
\begin{frame}{Resources}
%The \tttext{ipdw} R package
  \centering
  \url{http://cran.r-project.org/package=ipdw}
%   \begin{verbatim}
%   "install.packages("ipdw")"
%   \end{verbatim}
  \begin{thebibliography}{10}
  \beamertemplatearticlebibitems
  \bibitem{stachetal2016}
	Stachelek~J.,C.~J.~Madden,S.~P.~Kelly,M.~Blaha (submitted). Fine-scale relationships between phytoplankton abundance and environmental drivers in Florida Bay, USA.
	\newblock \doublequoted{Estuaries and Coasts}
  \bibitem{stachmadden2015}
	Stachelek~J.,C.~J.~Madden. 2015. Application of Inverse Path Distance Weighting for high-density spatial mapping of coastal water quality patterns
	\newblock \doublequoted{Int. J. Geographical Information Science}
  \end{thebibliography}

	\begin{itemize}
		\item \sout{\url{jstachel@sfwmd.gov}} \url{stachel2@msu.edu}
	\end{itemize}
\end{frame}

\end{document}